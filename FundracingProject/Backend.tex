\documentclass[12pt]{article}
\usepackage[left=12mm,top=0.5in,bottom=5in]{geometry}
\begin{document} 
\title{Backend}
\maketitle

\section {Classe}
La classe che si occuperà delle funzionalità di backend del progetto FundRaising Projec è DepositoDati.

\section {Funzioni prima del login}

La prima delle funzionalità che le viene richiesta,appena si apre la applicazione,è la verifica della correttezza dell'username e della password specificati nei campi appositi.\\
In contemporanea,vengono prelevati e inseriti nella tabella tutti i progetti con uno stake e un progresso momentaneamente indefiniti.\\

\section {Funzioni automatiche dopo il  login}
Se le verifica ha esito positivo,vengono presi dal database i dati riguardanti l'azienda: l'indirizzo,il nome,il cap e l'immagine.\\
I progetti all'interno della tabella vengono aggiornati,questa volta con il reale progress e stake,caricandoli nel seguente ordine: \\
1) I progetti proprietari dell'azienda appena loggata \\
2)I progetti su cui l'azienda ha fatto almeno un finanziamento \\
3)Tutti i progetti rimanenti,che non rientrano nelle prime due categorie.\\

\section {Funzioni disponibili dopo il login}

La classe DepositoDati \\
1) quando si esegue un updateStake,inserisce un nuovo finanziamento corrispondente alla differenza tra il nuovo e il vecchio stake \\
2) quando si inserisce un progetto,crea la row ad esso associata nella tabella "progetto" e il primo finanziamento,con stake 0,nella omonima tabella \\
3)  quando si clicca il bottone delete,dopo aver selezionato una riga della tabella,prima verifica se si è il proprietario del progetto selezionato,e se l'esito della verifica è positivo,elimina il progetto e tutti i finanziamenti associati dal database,altrimenti solo il finanziamento,qualora sia mio

\section {Funzioni ausiliari}

Ci sono infine delle funzioni ausiliarie necessarie per il corretto funzionamento della applicazione:\\
1)Recuperare dal database la descrizione e lo stake del progetto associato alla riga della tabella cliccata\\
2)Verifica se si è proprietari di un progetto\\
3)Recupero dell'id dell'ultimo progetto inserito\\

\end{document}